\let\negmedspace\undefined{}
\let\negthickspace\undefined{}
\documentclass[journal,12pt,twocolumn]{IEEEtran}
 \usepackage{gensymb}
 \usepackage{polynom}
\usepackage{amssymb}
\usepackage[cmex10]{amsmath}
 \usepackage{amsthm}
  \usepackage{stfloats}
\usepackage{bm} 
 \usepackage{longtable}
 \usepackage{enumitem}
 \usepackage{mathtools}
 \usepackage{tikz}
   \usepackage[breaklinks=true]{hyperref}
 \usepackage{listings}
\usepackage{color}                                            
\usepackage{array}                                            
\usepackage{longtable}                                        
\usepackage{calc}                                             
     \usepackage{multirow}                                         
     \usepackage{hhline}                                           
     \usepackage{ifthen}                                           
     \usepackage{lscape}     
\DeclareMathOperator*{\Res}{Res}
\DeclareMathOperator*{\equals}{=}
\renewcommand\thesection{\arabic{section}}
\renewcommand\thesubsection{\thesection.\arabic{subsection}}
\renewcommand\thesubsubsection{\thesubsection.\arabic{subsubsection}}
\renewcommand\thesectiondis{\arabic{section}}
\renewcommand\thesubsectiondis{\thesectiondis.\arabic{subsection}}
\renewcommand\thesubsubsectiondis{\thesubsectiondis.\arabic{subsubsection}}
\hyphenation{op-tical net-works semi-conduc-tor}
 \def\inputGnumericTable{}                                 %%
\lstset{ 
frame=single,
breaklines=true,
columns=fullflexible
}
\begin{document}
\newtheorem{theorem}{Theorem}[section]
\newtheorem{problem}{Problem}
\newtheorem{proposition}{Proposition}[section]
\newtheorem{lemma}{Lemma}[section]
\newtheorem{corollary}[theorem]{Corollary}
\newtheorem{example}{Example}[section]
\newtheorem{definition}[problem]{Definition}
\newcommand{\BEQA}{\begin{eqnarray}}
\newcommand{\EEQA}{\end{eqnarray}}
\newcommand{\define}{\stackrel{\triangle}{=}}
\newcommand*\circled[1]{\tikz[baseline= (char.base)]{
    \node[shape=circle,draw,inner sep=2pt] (char) {#1};}}
\bibliographystyle{IEEEtran}
\providecommand{\mbf}{\mathbf}
\providecommand{\pr}[1]{\ensuremath{\Pr\left(#1\right)}}
\providecommand{\qfunc}[1]{\ensuremath{Q\left(#1\right)}}
\providecommand{\sbrak}[1]{\ensuremath{{}\left[#1\right]}}
\providecommand{\lsbrak}[1]{\ensuremath{{}\left[#1\right.]}}
\providecommand{\rsbrak}[1]{\ensuremath{{}\left[#1\right.]}}
\providecommand{\brak}[1]{\ensuremath{\left(#1\right)}}
\providecommand{\lbrak}[1]{\ensuremath{\left(#1\right.)}
\providecommand{\rbrak}[1]{\ensuremath{\left[#1\right.]}}}
\providecommand{\cbrak}[1]{\ensuremath{\left\{#1\right\}}}
\providecommand{\lcbrak}[1]{\ensuremath{\left\{#1\right.}}
\providecommand{\rcbrak}[1]{\ensuremath{\left.#1\right\}}}
\theoremstyle{remark}
\newtheorem{rem}{Remark}
\newcommand{\sgn}{\mathop{\mathrm{sgn}}}
\providecommand{\abs}[1]{\left\vert#1\right\vert}
\providecommand{\res}[1]{\Res\displaylimits_{#1}} 
\providecommand{\norm}[1]{\left\lVert#1\right\rVert}
\providecommand{\mtx}[1]{\mathbf{#1}}
\providecommand{\mean}[1]{E\left[ #1 \right]}
\providecommand{\fourier}{\overset{\mathcal{F}}{ \rightleftharpoons}}
\providecommand{\system}{\overset{\mathcal{H}}{ \longleftrightarrow}}
\newcommand{\solution}{\noindent \textbf{Solution: }}
\newcommand{\cosec}{\,\text{cosec}\,}
\newcommand*{\permcomb}[4][0mu]{{{}^{#3}\mkern#1#2_{#4}}}
\newcommand*{\perm}[1][-3mu]{\permcomb[#1]{P}}
\newcommand*{\comb}[1][-1mu]{\permcomb[#1]{C}}
\renewcommand{\thetable}{\arabic{table}} 
\providecommand{\dec}[2]{\ensuremath{\overset{#1}{\underset{#2}{\gtrless}}}}
\newcommand{\myvec}[1]{\ensuremath{\begin{pmatrix}#1\end{pmatrix}}}
\newcommand{\mydet}[1]{\ensuremath{\begin{vmatrix}#1\end{vmatrix}}}
\numberwithin{equation}{section}
\numberwithin{figure}{section}
\numberwithin{table}{section}
\makeatletter
\@addtoreset{figure}{problem}
\makeatother
\let\StandardTheFigure\thefigure{}
\let\vec\mathbf{}
\def\putbox#1#2#3{\makebox[0in][l]{\makebox[#1][l]{}\raisebox{\baselineskip}[0in][0in]{\raisebox{#2}[0in][0in]{#3}}}}
     \def\rightbox#1{\makebox[0in][r]{#1}}
     \def\centbox#1{\makebox[0in]{#1}}
     \def\topbox#1{\raisebox{-\baselineskip}[0in][0in]{#1}}
     \def\midbox#1{\raisebox{-0.5\baselineskip}[0in][0in]{#1}}
\vspace{3cm}
\title{Assignment 6 12th Class}
\author{Gunjit Mittal (AI21BTECH11011)}
\maketitle
Download all python codes from 
\begin{lstlisting}
https://github.com/GunjitMittal/Assignment6/tree/main/Assignment6/code
\end{lstlisting}
Download all latex codes from 
\begin{lstlisting}
https://github.com/GunjitMittal/Assignment6/tree/main/Assignment6 
\end{lstlisting} 
\section{Question}
Let a pair of dice be thrown and the random variable X be the sum of the
numbers that appear on the two dice. Find the mean or expectation of X.
\section{Solution} 
\solution{}
The sample space of the experiment consists of 36 elementary events in the
form of ordered pairs ($x_i, y_i$), where $x_i $= 1, 2, 3, 4, 5, 6 and $y_i$ = 1, 2, 3, 4, 5, 6.\\
The random variable X i.e.\ the sum of the numbers on the two dice takes the
values 2, 3, 4, 5, 6, 7, 8, 9, 10, 11 or 12.\\
Now
\begin{align}
&\Pr(X = 2) = \Pr({(1,1)}) = \frac{1}{36}\\
&\Pr(X = 3) = \Pr({(1,2), (2,1)}) = \frac{2}{36}\\ 
&\Pr(X = 4) = \Pr({(1,3), (2,2), (3,1)}) = \frac{3}{36}\\
&\Pr(X = 5) = \Pr({(1,4), (2,3), (3,2), (4,1)}) \nonumber\\ 
&~~~~~~~~~~~~~~~~~~~~~~~~~~~~~~~~~~~~~~~~= \frac{4}{36}\\
&\Pr(X = 6) = \Pr((1,5), (2,4), (3,3), (4,2), \nonumber\\
& ~~~~~~~~~~~~~~~~~~~~~~~~~~~~~~~~~~~~~(5,1)) = \frac{5}{36}\\
&\Pr(X = 7) = \Pr((1,6), (2,5), (3,4), (4,3),\nonumber\\
&~~~~~~~~~~~~~~~~~~~~~~~~~~~~~(5,2), (6,1)) = \frac{6}{36}
\end{align}
\begin{align}   
&\Pr(X = 8) = \Pr((2,6), (3,5), (4,4), (5,3),\nonumber\\ 
&~~~~~~~~~~~~~~~~~~~~~~~~~~~~~~~~~~~~~~(6,2)) = \frac{5}{36}\\   
&\Pr(X = 9) = \Pr({(3,6), (4,5), (5,4), (6,3)}) \nonumber\\
&~~~~~~~~~~~~~~~~~~~~~~~~~~~~~~~~~~~~~~~~~~= \frac{4}{36}\\ 
&\Pr(X = 10) = \Pr({(4,6), (5,5), (6,4)}) = \frac{3}{36}\\
&\Pr(X = 11) = \Pr({(5,6), (6,5)}) = \frac{2}{36}\\  
&\Pr(X = 12) = \Pr({(6,6)}) = \frac{1}{36} 
\end{align}    
The probability distribution of X is  
\begin{table}[h]  
    \addtolength{\tabcolsep}{-5pt}\small
    \input{tables/table6.tex}\label{Table 1}	
\end{table} 
Therefore,\\
\begin{align}
    &\mu = E(x) = \sum_{i=1}^{n}{x_i}{p_i}  = 2\times\frac{1}{36}+3\times\frac{2}{36}+4\times\frac{3}{36}+\nonumber\\
    &~~~~~5\times\frac{4}{36}+6\times\frac{5}{36}+7\times\frac{6}{36}+8\times\frac{5}{36}+9\times\frac{4}{36}+\nonumber\\
    &~~~~~~~~~~~~~~~~10\times\frac{3}{36}11\times\frac{2}{36}+12\times\frac{1}{36}\\
    &=\frac{2+6+12+20+30+42+40+36+30+22+12}{36}\\
    &~~~~~~~~~~~~~~~~~~~~~~~~~~= 7
\end{align}
Thus, the mean of the sum of the numbers that appear on throwing two fair dice is 7.
\end{document}   